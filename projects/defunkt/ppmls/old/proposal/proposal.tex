\documentclass[a4]{article}
\usepackage{epsfig}
\usepackage[polish,english]{babel}

\textwidth=7in \textheight=9.3in \oddsidemargin=-0.35in
\headheight=-1in


% begin Ela -------------------------------------------------
% this is to print a Polish letter correctly in LaTeX
\newdimen\plWleft
\newdimen\plWdown
\newdimen\plWright
\newdimen\plWtemp
\def\sob#1#2#3#4#5{%parameters: letter and fractions hl,ho,vl,vo
  \setbox0\hbox{#1}\setbox1\hbox{$_\mathchar'454$}\setbox2\hbox{p}%
  \plWright=#2\wd0 \advance\plWright by-#3\wd1
  \plWdown=#5\ht1 \advance\plWdown by-#4\ht0
  \plWleft=\plWright \advance\plWleft by\wd1
  \plWtemp=-\plWdown \advance\plWtemp by\dp2 \dp1=\plWtemp
  \leavevmode
  \kern\plWright\lower\plWdown\box1\kern-\plWleft #1}
\DeclareTextCommand{\e}{OT1}{\sob e{.50}{.35}{0}{.93}}
% end Ela -------------------------------------------------



\title{A Workshop Proposal for ECML 2007:\\
\bf Principles and Practice of Multiple Learning Systems}
\author{Dr Gavin Brown, Dr El\.zbieta P\e{}kalska\\University of Manchester, UK}
\date{}

\begin{document}
\maketitle

\noindent{\bf A brief technical description of the specific
technical issues that the workshop will address.}\\
The workshop aims to be representative of the current
state-of-the-art in Multiple Learning Systems.  This encompasses not
only the well-known classifier ensembles idea, but also the broader
concept of combining learning {\em methodologies}.  We aim to attract papers
addressing theory/practice in particularly challenging domains such
as non-stationarity domains and massive online stream mining, and
the latest theoretical questions including discriminative/generative
learning, probabilistic/information-theoretic learning, and kernel
combining methods.\\
\\
{\bf The reasons why the workshop is of interest this time.}\\
The combining paradigm has had a profound impact on pattern
recognition over the last decade, of interest to a wide community
from theorists in Machine Learning to practitioners in Data Mining.
This will be the first ECML workshop to address the issue
explicitly, incorporating applications of current methods, open
theoretical questions, priorities for different application domains,
and advanced aspects of the \emph{combining principle} such as the
compatability of different
learning \emph{algorithms}.\\
\\
\noindent{\bf The names, postal and email addresses, phone and fax
numbers of
the Workshop Chairs.}\\
Gavin Brown and El\.zbieta P\e{}kalska,\\[1mm]
\emph{gbrown@cs.man.ac.uk, pekalska@cs.man.ac.uk}\\
School of Computer Science\\
University of Manchester\\
Oxford Road\\
Manchester M13 9PL, UK\\
Tel. (Brown): +44 (0)161 275 6190\\
Tel. (P\e{}kalska): +44 (0)161 275 6194\\
Fax (Both): 0161 275 6204\\
\\
{\bf The names, affiliations and email addresses of the Workshop
Program Committee}
\begin{itemize}
\setlength{\itemsep}{-1mm}
\item{  Robert Duin (Delft, Netherlands), r.duin@ieee.org}
\item{  Giorgio Fumera (Calgiari, Sardinia), fumera@diee.unica.it}
\item{  Tin Kam Ho (Bell Labs, USA), tkh@research.bell-labs.com}
\item{  Tim Kovacs (Bristol, UK), kovacs@cs.bris.ac.uk}
\item{  Natalio Krasnogor (Nottingham, UK), natalio.krasnogor@nottingham.ac.uk}
\item{  Ludmila Kuncheva (Bangor, Wales), l.i.kuncheva@bangor.ac.uk}
\item{  Nikunj Oza (NASA Ames, USA), oza@email.arc.nasa.gov}
\item{  Fabio Roli (Calgiari, Sardinia), roli@diee.unica.it}
\item{  Amanda Sharkey (Sheffield, UK), sharkey@dcs.shef.ac.uk}
\item{  Marina Skurichina (Delft, Netherlands), m.skurichina@tudelft.nl}
\item{  Terry Windeatt (Surrey, UK), t.windeatt@surrey.ac.uk}
\item{  Jeremy Wyatt (Birmingham, UK), jlw@cs.bham.ac.uk}
\end{itemize}
(We have contacted all of the above and all have agreed to serve.)\\
\\
\\
\noindent{\bf A list of previously-organized related workshops by any of the
Workshop Chairs.}\\
{\em Brown}: Workshop on Ensemble Theory and Practice, University of Birmingham,
November 2004\\
\\
{\bf How the workshop will be publicized.}\\
Emails will be sent to the Connectionists mailing list, the Multiple Classifier Systems mailing list, and 
several individual research groups at the institutions of the programme committee members.\\
\\
{\bf A summary of the intended workshop Call for Participation.}\\
Attached to this application, see also the preliminary workshop website at:\\

{\tt http://www.cs.man.ac.uk/\~{}gbrown/ppmls/}\\
\\
We have secured Dr Ludmila Kuncheva as invited speaker for the workshop.  Dr Kuncheva is widely recognised
as a leading figure in the field, having authored over 70 publications on the topic,
including the central textbook, "Combining Pattern Classifiers".  We will also organise a poster session, 
encouraging unfinished or contraversial work that may invite debate. {\em \bf We also intend to propose
a tutorial on this topic if the workshop is of interest to the ECML chairs.}\\
\\
\noindent{\bf Plans to document the workshop results (beyond ECML/PKDDS publication).}\\
Proceedings with be published online and provided to participants.\\
Oza (on our program committee) is an associate editor with the Journal of 
Information Fusion, and has confirmed that given sufficient submissions, a special issue on this topic is
of interest at this time.\\
\\
\noindent{\bf Ideal duration of the workshop (three/ four sessions).}\\
Three sessions.\\
\\
\noindent{\bf A list of audio-visual or technical requirements and any special room requirements.}\\
A data projector and sufficent space/equipment to organise a poster session if it is of interest to the ECML 
chairs.


\end{document}
